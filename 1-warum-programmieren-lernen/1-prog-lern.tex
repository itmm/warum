\documentclass[a5paper,ngerman]{article}
\usepackage[T1]{fontenc}
\usepackage[utf8]{inputenc}
\usepackage[margin=0.5in,includefoot]{geometry}
\usepackage{microtype}
\usepackage{babel}
\usepackage{siunitx}
\usepackage{solarized}
\usepackage{sectsty}
\usepackage{ccfonts}
\usepackage{euler}
\usepackage{fancyhdr}
\usepackage{fancyvrb}
\usepackage{graphicx}
\usepackage{multicol}
\usepackage{mdframed}
\usepackage{lisp}
\pagestyle{fancy}
\fancypagestyle{plain}{
\fancyhf{}
\fancyfoot[C]{{\color{deemph}\small$\thepage$}}
\renewcommand{\headrulewidth}{0pt}
\renewcommand{\footrulewidth}{0pt}}
\title{\color{emph}Warum sollten wir alle programmieren können?}
\author{Timm Knape}
\date{\today}
\columnseprule.2pt
\renewcommand{\columnseprulecolor}{\color{deemph}}
\begin{document}
\pagecolor{background}
\color{normal}
\allsectionsfont{\color{emph}\mdseries}
\pagestyle{plain}
\maketitle
\thispagestyle{fancy}
\surroundwithmdframed[backgroundcolor=codebackground,fontcolor=normal,hidealllines=true]{Verbatim}
\begin{multicols}{2}

\end{multicols}
\section{Computer sind kompliziert}
\begin{multicols}{2}

"Any sufficiently advanced technology is indistinguishable from magic."
(Arthur C. Clarke)

"There's a good part of Computer Science that's like magic.
Unfortunately there's a bad part of Computer Science that's like
religion." (Hal Ableson)

\end{multicols}
\section{Wer steuer wen?}
\begin{multicols}{2}

\end{multicols}
\section{Verantwortung}
\begin{multicols}{2}

Kann ein Computer eine Tat verantworten? Vermutlich nicht.

Die Bank-Mitarbeiterin zuckt nur mit den Schultern:
"Tut mir leid, wir können Ihnen den Haus-Kredit nicht bewilligen: der
Rechner hat den Kredit-Antrag abgelehnt."

Wie kann denn ein Rechner etwas ablehnen?
Wir fragen ja nicht den Rechner, ob er uns Geld leiht, sondern die
Bank.
Also ist es auch die Bank, die den Antrag abgelehnt hat.

Es ist auch nicht die Mitarbeiterin, zumindest wollen wir das für sie
jetzt mal hoffen.
Vielmehr wurde der Mitarbeiterin vermutlich mitgeteilt, dass sie
Anträge mit irgendeinem Programm prüfen muss.
Wenn das Programm abschlägig antwortet, dann darf sie den Antrag nicht
genehmigen.

Für die Bank-Angestellte ist die Situation leicht.
Sie lehnt den Antrag ab, weil sie ihn ablehnen soll.
Und es wäre so einfach, wenn wir uns damit zufrieden geben würden.

Stellen wir nun die interessante Frage:
*Warum wurde der Antrag abgelehnt?*

Mehrere Antworten würden uns zufrieden stellen:

\begin{itemize}
\item Weil bei Ihrem Einkommen nicht genügend Reserven für Zinsen und
  Tilgung übrig bleiben. Wir haben keine Ahnung, wie sie überhaupt so
  lange überlebt haben.
\item Da sie schon zweimal wegen Kreditbetrugs verurteilt wurden, ist uns
  das Risiko eines Ausfalls zu hoch.
\item Sie sind nicht unser Ziel-Publikum. Wir finanzieren nur Projekte im
  sieben- bis achtstelligen Bereich.
\end{itemize}

Aber vermutlich sieht die Antwort eher so aus:

\begin{itemize}
\item Weil der Computer das entschieden hat.
\end{itemize}

Wenn wir anmerken, dass ein Computer nicht \emph{entscheiden}\ sondern nur
Sachen \emph{berechnen}\ kann, wechselt die Antwort vielleicht zu

\begin{itemize}
\item Weil der Computer das berechnet hat.
\end{itemize}

Diese Aussagen sind nicht zufrieden stellend.
Aber der Angestellten ist kein Vorwurf zu machen: sie weiß es
vermutlich gar nicht besser.
Der Auftrag war ja lediglich: wenn der Computer "nein" sagt, dann
bloß keinen Kredit vergeben.


\end{multicols}
\end{document}
