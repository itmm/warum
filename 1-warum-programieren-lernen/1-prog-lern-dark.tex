\documentclass[a5paper,ngerman]{article}
\usepackage[T1]{fontenc}
\usepackage[utf8]{inputenc}
\usepackage[margin=0.5in,includefoot]{geometry}
\usepackage{microtype}
\usepackage{babel}
\usepackage{siunitx}
\usepackage[dark]{solarized}
\usepackage{sectsty}
\usepackage{ccfonts}
\usepackage{euler}
\usepackage{fancyhdr}
\usepackage{fancyvrb}
\usepackage{graphicx}
\usepackage{multicol}
\usepackage{mdframed}
\usepackage{lisp}
\pagestyle{fancy}
\fancypagestyle{plain}{
\fancyhf{}
\fancyfoot[C]{{\color{deemph}\small$\thepage$}}
\renewcommand{\headrulewidth}{0pt}
\renewcommand{\footrulewidth}{0pt}}
\title{\color{emph}Warum soll ich programmieren lernen?}
\author{Timm Knape}
\date{\today}
\columnseprule.2pt
\renewcommand{\columnseprulecolor}{\color{deemph}}
\begin{document}
\pagecolor{background}
\color{normal}
\allsectionsfont{\color{emph}\mdseries}
\pagestyle{plain}
\maketitle
\thispagestyle{fancy}
\surroundwithmdframed[backgroundcolor=codebackground,fontcolor=normal,hidealllines=true]{Verbatim}
\begin{multicols}{2}

\end{multicols}
\section{Was ist Programmieren?}
\begin{multicols}{2}

Fangen wir vorne an:

\textbf{Programmieren}\ nennt man die Tätigkeit, Programme für Computer zu
erstellen.

Scheinbar ist mit dieser Definition nicht viel gewonnen.
Viel mehr müssen wir erst einmal klären, was ein Programm und ein
Computer ist, bevor die obige Aussage anfängt, einen Sinn zu bekommen.

Fangen wir mit dem konkreteren Ding an:

\end{multicols}
\section{Was ist ein Computer?}
\begin{multicols}{2}

Fassen wir den Begriff des Rechners möglichst breit:

Ein \textbf{Rechner}\ ist eine Maschine, die Programme ausführt.

Immer noch nicht viel besser, oder?

Vielleicht wird es klarer, wenn wir die letzte Unbekannte beschreiben:

\end{multicols}
\section{Was ist ein Programm?}
\begin{multicols}{2}

\end{multicols}
\section{Warum braucht ein Rechner ein Programm?}
\begin{multicols}{2}

\end{multicols}
\section{Wo kommen die Programme her?}
\begin{multicols}{2}

\end{multicols}
\section{Eine Welt voller Programme}
\end{document}
